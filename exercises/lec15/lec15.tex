\documentclass[a4paper]{article}
\usepackage{xeCJK}
\usepackage[margin=1in]{geometry}
\usepackage{amsmath}
\usepackage{amsthm}
\usepackage{amsfonts}
\usepackage{bbm}
\usepackage{parskip}
\usepackage{listings}
\setlength{\parindent}{0cm}
\usepackage{graphicx}
\usepackage{float}
\usepackage{array}
\usepackage{multirow}
\newcommand*\diff{\mathop{}\!\mathrm{d}}
 \newcommand{\sgn}{\operatorname{sgn}}
\renewcommand{\arraystretch}{1.2}
\newcolumntype{C}[1]{>{\centering\let\newline\\\arraybackslash\hspace{0pt}}m{#1}}

\title{Lec 15 Exercise}
\author{
        计44 张欣阳 2014011561
}
\date{2017-04-19}

\begin{document}

\maketitle

\section*{题目1}
证明短进程优先算法具有最小平均周转时间

\begin{proof}[\textbf{证明}]
    对于调度序列$r_1, r_2, ..., r_n$,我们称$(r_i, r_j)$为一个逆序对,如果$i < j$且进程运行时间$r_i \beq r_j$,我们先证明调换$(r_i, r_j)$平均周转时间将会变短。

    调换后,平均周转时间的变化是:
    \begin{align*}
        \Delta t &= \frac{1}{n}[(n-i+1)*r_j + (n-j+1)*r_i - (n-i+1)*r_i - (n-j+1)*r_j]\\
        &= \frac{j-i}{n}(r_j-r_i)
    \end{align*}
    因为$i<j$,且$r_i \beq r_j$,所以$\Delta t \leq 0$,即平均周转时间变短了。所以对于一个序列,如果有逆序对存在,总可以通过交换逆序对的方式使平均周转时间变短。那么对于短进程优先算法,由于其调度序列没有逆序对,所以有最小平均周转时间。

\end{proof}


\end{document}
